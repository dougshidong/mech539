\documentclass[letterpaper,12pt,]{article}

\usepackage{titling}

\setlength{\droptitle}{5in}   % This is your set screw

\usepackage[%
    left=1in,%
    right=1in,%
    top=1in,%
    bottom=1.0in,%
    paperheight=11in,%
    paperwidth=8.5in%
]{geometry}%
\usepackage{comment}

\usepackage{listings}
\usepackage{graphicx}
\usepackage{amsmath}
\usepackage[section]{placeins}
\usepackage[font=small,skip=0pt]{caption}
\usepackage{subcaption}
\usepackage{hyperref}
\usepackage{booktabs}
\usepackage{pdfpages}


\lstdefinestyle{mystyle}{
    %backgroundcolor=\color{backcolour},
    %commentstyle=\color{codegreen},
    %keywordstyle=\color{magenta},
    %numberstyle=\tiny\color{codegray},
    %stringstyle=\color{codepurple},
    basicstyle=\footnotesize,
    breakatwhitespace=false,
    breaklines=true,
    captionpos=b,
    keepspaces=true,
    numbers=left,
    numberstyle=\footnotesize,
    stepnumber=1,
    numbersep=5pt,
    showspaces=false,
    showstringspaces=false,
    showtabs=false,
    tabsize=2,
    frame=single
}
\lstset{frame=single}

\pagestyle{empty} % Remove page numbering
\linespread{1.5} % Line Spacing

\begin{document}

\input{titlepage.tex}

\section*{Question 1 \& 2}

\begin{figure}[!h]
    \centering
    \includegraphics[width = 0.95\textwidth]{./figures/q1pressure.pdf}
    \caption{Pressure Distribution NACA0012, $Re = 1\textsc{e}06$, $\alpha = 4^{\circ}$}
    \label{fig:q1p}
\end{figure}

The pressure distribution for various number of panels is shown in Figure \ref{fig:q1p}.
As the number of panels increase, the pressure distribution converges to a smoother shape.
It is possible to visually differentiate the pressure distribution of 20, 30 and 40 panels from the other results.

The lift and drag coefficients are compiled in Table \ref{tab1}.
Their convergence are graphically shown in Figure \ref{fig:q1l} and \ref{fig:q1d}.
As the number of panels increases, the lift and drag coefficients also converge.
For lower numbers of panels, the drag coefficient is higher than its converged value and the lift coefficient is lower than its converged value.
Since less panels entails a bigger grid spacing, the truncation error is higher for coarser grids.
Numerical dissipation has stronger effects for coarse grids, resulting in weaker performances of the airfoil.

From Figure \ref{fig:q1l} and \ref{fig:q1d}, 30-40 panels seem to be sufficient to get a reasonably accurate lift coefficient, while 50 panels are required to get an accurate drag coefficient.

\begin{table}[!h]
\centering
\begin{tabular}{ccc} \toprule
    {Number of Panels} & {$C_L$} & {$C_D$} \\ \midrule
    {20} & 0.4488  & 0.0093\\
    {30} & 0.4673  & 0.0095\\
    {40} & 0.4724  & 0.0095\\
    {50} & 0.4746  & 0.0084\\
    {60} & 0.4759  & 0.0083\\
    {70} & 0.4768  & 0.0083\\
    {80} & 0.4774  & 0.0082\\
    {90} & 0.4778  & 0.0083\\
    {100} & 0.4782  & 0.0082 \\
\bottomrule
\end{tabular}
\caption{Lift and Drag Coeffcients for NACA0012, $Re = 1\textsc{e}06$, $\alpha = 4^{\circ}$}
\label{tab1}
\end{table}

\begin{figure}[!h]
    \centering
    \includegraphics[width = 0.95\textwidth]{./figures/q1lift.pdf}
    \caption{Lift Grid Study}
    \label{fig:q1l}
\end{figure}

\begin{figure}[!h]
    \centering
    \includegraphics[width = 0.95\textwidth]{./figures/q1drag.pdf}
    \caption{Drag Grid Study}
    \label{fig:q1d}
\end{figure}

\section*{Question 3}



\section*{Question 4}



\section*{Question 7}


\section*{Codes}

Code has been written in FORTRAN. Default arithmetic operations are in double precision and optimization level -O3 unless specified otherwise.

All codes are available on my GitHub:

\url{https://github.com/dougshidong/mech539/tree/master/a2}

\end{document}
