\documentclass[letterpaper,12pt,]{article}

\usepackage{titling}

\setlength{\droptitle}{5in}   % This is your set screw

\usepackage[%
    left=1in,%
    right=1in,%
    top=1in,%
    bottom=1.0in,%
    paperheight=11in,%
    paperwidth=8.5in%
]{geometry}%
\usepackage{comment}

\usepackage{listings}
\usepackage{graphicx}
\usepackage{amsmath}
\usepackage[section]{placeins}
\usepackage[font=small,skip=-2pt]{caption}
\usepackage{subcaption}
\usepackage{hyperref}
\usepackage{booktabs}
\usepackage{pdfpages}
\usepackage{pgffor}


\lstdefinestyle{mystyle}{
    %backgroundcolor=\color{backcolour},
    %commentstyle=\color{codegreen},
    %keywordstyle=\color{magenta},
    %numberstyle=\tiny\color{codegray},
    %stringstyle=\color{codepurple},
    basicstyle=\footnotesize,
    breakatwhitespace=false,
    breaklines=true,
    captionpos=b,
    keepspaces=true,
    numbers=left,
    numberstyle=\footnotesize,
    stepnumber=1,
    numbersep=5pt,
    showspaces=false,
    showstringspaces=false,
    showtabs=false,
    tabsize=2,
    frame=single
}
\lstset{frame=single}

\pagestyle{empty} % Remove page numbering
\linespread{1.5} % Line Spacing

\begin{document}

\input{titlepage.tex}


\section*{Grid}

The coarse grid has $80$ nodes in the $x$-direction and $40$ nodes in the $y$-direction with a constant spacing $dx = 0.025$ along the airfoil and $dy=0.025$ on the first layer in the $y$-direction.
The grid spans $x=[-0.05, 47.5]$ and $y=[0.0, 50.9]$.

When a finer grid is required, additional points are inserted evenly between each node to refine the grid.
The $159 \times 79$ and $317 \times 157$ grids will be refered as the medium and fine grids respectively.

\section*{Mach Number Study}

In this section, the coarse grid is used and the freestream Mach number $M_{in}$ is varied between $[0.80,0.90]$ with increments of $0.02$.
Figure \ref{fig:conv2} shows the convergence of the algorithm down to 5\textsc{E-}15.
Higher $M_{in}$ seem to take slightly more iterations to converge.
It can be explained by looking at the earlier iterations shown in Figure \ref{fig:conv22}.
The discontinuous jumps in residual are probably due to the sonic switches in the Murman-Cole algorithm turning on and off.
Higher $M_{in}$ (in the transonic region) will result in more inaccurate initial conditions and therefore, more turning on and off the switches.

The surface pressure coefficient is shown in Figure \ref{fig:cpsurf}.
As the $M_{in}$ increases, a shock forms and starts moving aft, until it reaches the trailing edge.
Moreover, the strength of the shock also increases since a higher decrease in static pressure is required to retrieve the atmosphere pressure.
Figure \ref{fig:cpcontour8084} and \ref{fig:cpcontour8690} show the coefficient of pressure contours for the different $M_{in}$.

\begin{figure}[!h]
    \centering
    \includegraphics[width = 0.85\textwidth]{./Figures/convergenceq2.pdf}
    \caption{Convergence of Murman-Cole Algorithm}
    \label{fig:conv2}
\end{figure}

\begin{figure}[!h]
    \centering
    \includegraphics[width = 0.85\textwidth]{./Figures/convergenceq22.pdf}
    \caption{Convergence of Murman-Cole Algorithm}
    \label{fig:conv22}
\end{figure}

\begin{figure}[!h]
    \centering
    \includegraphics[width = 0.85\textwidth]{./Figures/cpsurf.pdf}
    \caption{$C_p$ Surface Distribution}
    \label{fig:cpsurf}
\end{figure}

\begin{figure}
\centering
\foreach \i in {80,82,84} {%
    \begin{subfigure}[p]{0.83\textwidth}
        \includegraphics[width=\linewidth]{./Figures/cpcontour_\i.pdf}
    \end{subfigure}
}
\caption{$C_p$ Contour Plots for Mach = 0.80, 0.82, 0.84}
\label{fig:cpcontour8084}
\end{figure}

\begin{figure}
\centering
\foreach \i in {86,88,90} {%
    \begin{subfigure}[p]{0.83\textwidth}
        \includegraphics[width=\linewidth]{./Figures/cpcontour_\i.pdf}
    \end{subfigure}
}
\caption{$C_p$ Contour Plots for Mach = 0.86, 0.88, 0.90}
\label{fig:cpcontour8690}
\end{figure}

\section*{Codes}

Code has been written in FORTRAN. Default arithmetic operations are in double precision and optimization level -O3.

All codes are available on my GitHub:

\url{https://github.com/dougshidong/mech539/tree/master/a3}

\end{document}
